\documentclass[twoside,a4paper]{book}
\usepackage{graphicx}
\usepackage{hyperref}
\usepackage{amsmath}
\usepackage{amssymb}
\usepackage{textcomp}
\usepackage[utf8]{inputenc}
\usepackage[polish]{babel}
\usepackage[T1]{fontenc}
\usepackage{standalone}
\usepackage{array}
% pakiet stosowany do url'i w bibliografii, zamienia odnośniki na ładnie sformatowane
\usepackage{url}
% pakiety służące do numerowania i tworzenia algorytmów
\usepackage{algorithmic}
\usepackage{algorithm}
% redefinicja etykiety nagłówkowej listy algorytmów, domyślna jest po angielsku
\renewcommand{\listalgorithmname}{Spis algorytmów}

\usepackage[section]{placeins}
\usepackage{pdfpages}

% pakiet do wyliczania skali, przydatny przy dużych obrazkach
\usepackage{pgf}
% pakiet służący do automatycznego sortowania odnośników do bibliografii
\usepackage[sort]{natbib}
% tworzenie listingów
\usepackage{listings}
% tworzenie figur wewnątrz figur
\usepackage{subfig}
% do automatycznego skracania nazw rozdziałów i podrozdziałów używanych w nagłówkach strony by mieściły się w jednej linii
\usepackage[fit]{truncate}
% fancyhdr - ładne nagłówki, definicja wyglądu nagłówka, numery stron będą umieszczane w nagłówku po odpowiedniej stronie
\usepackage{fancyhdr}
\pagestyle{fancy}
\renewcommand{\chaptermark}[1]{\markboth{#1}{}}
\renewcommand{\sectionmark}[1]{\markright{\thesection\ #1}}



\fancyhf{}
\fancyhead[LE,RO]{\bfseries\thepage}
% tutaj ograniczamy szerokość pola w nagłówku zawierającego nazwę rozdziału/podrozdziału do 95% szerokości strony
% redefinicja sposobu prezentacji nazw domyślnie wypisywanych wielkimi literami (np. domyślnie w nagłówku Spis treści będzie miał postać SPIS TREŚCI)
% Uwaga! to może popsuć wielkie litery w ogóle! Jak coś nie działa należy usunąć \nouppercase{} z poniższych definicji
\fancyhead[LO]{\nouppercase{\bfseries{\truncate{.95\headwidth}{\rightmark}}}}
\fancyhead[RE]{\nouppercase{\bfseries{\truncate{.95\headwidth}{\leftmark}}}}
\renewcommand{\headrulewidth}{0.5pt}
\renewcommand{\footrulewidth}{0pt}

% definicja typu prostego wymagana przez pierwsze strony rozdziałów itp.
% powyższe reguły niestety tych stron nie dotyczą, gdyż Latex automatycznie przełącza je pomiędzy fancy a plain
% w tym wypadku eliminujemy nagłówki i stopki na stronach początkowych
\fancypagestyle{plain}{%
 \fancyhead{}
 \fancyfoot{}
 \renewcommand{\headrulewidth}{0pt}
 \renewcommand{\footrulewidth}{0pt}
}

\parskip 0.05in


% makro umożliwiające otaczanie symboli okręgami
\usepackage{tikz}
% brak justowania tekstu (bazą okręgu będzie linia tekstu)
\newcommand*\mycirc[1]{%
  \begin{tikzpicture}
    \node[draw,circle,inner sep=1pt] {#1};
  \end{tikzpicture}}

% pionowe justowanie tekstu, środek okręgu pokrywa się ze środkiem tekstu
\newcommand*\mycircalign[1]{%
  \begin{tikzpicture}[baseline=(C.base)]
    \node[draw,circle,inner sep=1pt](C) {#1};
  \end{tikzpicture}}

% zmiana nazwy twierdzeń i lematów
\newtheorem{theorem}{Twierdzenie}[section]
\newtheorem{lemma}[theorem]{Lemat}

% tworzenie definicji dowodu
\newenvironment{proof}[1][Dowód]{\begin{trivlist}
\item[\hskip \labelsep {\bfseries #1}]}{\end{trivlist}}
% \newenvironment{definition}[1][Definicja]{\begin{trivlist}
% \item[\hskip \labelsep {\bfseries #1}]}{\end{trivlist}}
% \newenvironment{example}[1][Przykład]{\begin{trivlist}
% \item[\hskip \labelsep {\bfseries #1}]}{\end{trivlist}}
% \newenvironment{remark}[1][Uwaga]{\begin{trivlist}
% \item[\hskip \labelsep {\bfseries #1}]}{\end{trivlist}}

% definicja czarnego prostokąta zwyczajowo dodawanego na koniec dowodu
\newcommand{\qed}{\nobreak \ifvmode \relax \else
      \ifdim\lastskip<1.5em \hskip-\lastskip
      \hskip1.5em plus0em minus0.5em \fi \nobreak
      \vrule height0.75em width0.5em depth0.25em\fi}

% poniższymi instrukcjami można sterować co ma być numerowane a co nie i co ma być wyświetlane w spisie treści
% \setcounter{secnumdepth}{3}
% \setcounter{tocdepth}{5}

% definicja czcionki mniejszej niż tiny (domyślnie takiej małej nie ma)
\usepackage{lmodern}
\makeatletter
  \newcommand\tinyv{\@setfontsize\tinyv{4pt}{6}}
\makeatother

% definicja jeszcze mniejszej czcionki
\usepackage{lmodern}
\makeatletter
  \newcommand\tinyvv{\@setfontsize\tinyvv{3.5pt}{6}}
\makeatother

% pakiet do obsługi wielostronicowych tabel
\usepackage{longtable}
\setlength{\LTcapwidth}{\textwidth}

\usepackage[section] {placeins}


\usepackage{multirow}

\usepackage{slantsc}
\usepackage[labelsep=endash]{caption}
\addto\captionspolish{\renewcommand{\figurename}{Rys.}}
\addto\captionspolish{\renewcommand{\tablename}{Tab.}}
\addto\captionspolish{\renewcommand*{\appendixpagename}{Dodatki}}
\addto\captionspolish{\renewcommand*{\appendixtocname}{Dodatki}}
\addto\captionspolish{\renewcommand*{\appendixname}{Dodatek}}

\setcounter{secnumdepth}{5}
\usepackage{listings}
\usepackage[toc,page]{appendix}


\begin{document}
\chapter{Podsumowanie}
Projekt ten z pozoru prosty natęczył wiele problemów natury algorytmicznej, które jednak udało się rozwiązać w wystarczająco dobry sposób, by móc powiedzieć o oprogramowaniu, że spełnia założone cele projektowe opisane w podrodziale ~\ref{sec:cele}, a co więcej pozostawia miejsce na jej dalszy rozwój. Poniżej pokrótce opisane zostaną wniki realizacji projektu oraz oraz propozycje jej rozwoju.
\section{Realizacja}
Jak wyżej wspomniano, udało się zrealizwać wszystkie zalożone wymagania oraz doprowadzić powstałą klawiaturę ekranową do tego stopnia użyteczności, że nawet wprowadzone zostały udogodnienia typu skrót klawiszowy otwierający menu kontekstowe oraz autouzupełnienie słów. Co więcej klawiatura jest w pełni kompatybilna z polskimi literami, czego nie można powiedzieć o dotychczasowych rozwiązaniach tego typu. Aplikacja jest komunikatywna zarówno poprzez broadcast z siecią lokalną jak i z siecią internetową.  Klawiatura jest konfigurowywalna oraz personalizowywalna.  
\subsection{Oprogramowanie}
Powstałe oprogramowanie stanowi spójną całość z gamą algorytmów opisanych w podrozdziale ~\ref{sec:algorytmy}. Algorytmy są opracowane do poziomu wystaczającego, jednak istnieje duża możliwość ich rozwoju i udoskonalenia w celu jeszcze lepszej pracy klawiatury. Więcej na temat możliwego rozwoju aplikacji poniżej.
\section{Możliwości dalszego rozwoju}
Po realizacji dotychczas zaplanowanych punktów zauważono, że istnieje szereg punktów, które można w przyszłości rozwinąć. Część opisano już w podrozdziale ~\ref{sec:whatMore}. 
\subsection{Interfejs klawiatury}
Należało by się zastanowić jak rozwiązać problem niewystarczającej przestrzeni ekranowej dla przycisków o wymiarach minimalnych 140px na 140px. Domyślnym rozwiązaniem byłoby zwiększenie dokładności sprzętu (okluarów) lub też zwiększenie minotra - co nie jest skutecznym rozwiązaniem dla chorych na stwardnienie boczne rozsiane. 

Udoskonalenia mogłyby wymagać również schematy kolorystyczne aplikacji. Wymagałoby to konsultacji z osobami chorymi, bądź osobami z ich otoczenia, czy też większego zasobu materiałów na temat optymalizacji wyglądów interfejsów oprogramowania. W wypadku oprogramownia dla chorych na ALS jest to bardzo znaczący czynnik, gdyż spędzać będą oni większość czasu na niego patrząc. Aplikacja będzie więc cześciej i dłużej wykorzystywana niż innego typu oprogramownia. Należy więc zadbać o zdrowotność interfejsu dla oczy pacjenta.  

Podążająć za opinią tesujących  opisaną w ~\ref{opinion}  przemyśleć należy również jak uczyniść interfejs bardziej intuicyjnym oraz jednocznacznym. 
\subsection{Algorytmy}
Odnosząc się również do punktu ~\ref{opinion} zdecydowanie należy rozwinąć mechanizm słownikowy, tak by ten działał na system wagowy tzn. sugerował się częstością użycia danego słowa, a dopiero w drugiej kolejności alfabetycznym porządkiem. Znaczącym ułatwieniem byłoby również automatyczne uzupełnianie słownika o nieistniejące słowa często używane, oraz możliwość ręcznego usunięcia słów nieużywanych. 

Dopracowania może wymagać również algorytm poruszania się po tekście.Należałoby umożliwić łatwe przewijanie tekstu w edytorze, jego zapisywanie oraz wczytywanie. Nacisk należy położyć na komunikację klawiatury z komputerem oraz programami zewnętrznymi.  Jeśli to byłoby możliwe, można by było zrezygować z dedykowanego API Google do wyszukiwania. 

Jednym z najważniejszych punktów wymagających jeszcze realizacji jest syntezator mowy, który symylowałaby wypowiadanie aktualnie wpisywanego tekstu. Aktulnie możliwość komunikacji jest zastąpiona przez wiadomości broadcast, jednak w celu symulacji naturalnej konwersacji syntezator ten jest niezbędnym czynnikiem. Bez niego konwersacje za pomocą klawiatry ekranowej są powolne i nienaturalne.
 


\end{document}