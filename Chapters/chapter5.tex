\documentclass[twoside,a4paper]{book}
\usepackage{graphicx}
\usepackage{hyperref}
\usepackage{amsmath}
\usepackage{amssymb}
\usepackage{textcomp}
\usepackage[utf8]{inputenc}
\usepackage[polish]{babel}
\usepackage[T1]{fontenc}
\usepackage{standalone}
\usepackage{array}
% pakiet stosowany do url'i w bibliografii, zamienia odnośniki na ładnie sformatowane
\usepackage{url}
% pakiety służące do numerowania i tworzenia algorytmów
\usepackage{algorithmic}
\usepackage{algorithm}
% redefinicja etykiety nagłówkowej listy algorytmów, domyślna jest po angielsku
\renewcommand{\listalgorithmname}{Spis algorytmów}

\usepackage[section]{placeins}
\usepackage{pdfpages}

% pakiet do wyliczania skali, przydatny przy dużych obrazkach
\usepackage{pgf}
% pakiet służący do automatycznego sortowania odnośników do bibliografii
\usepackage[sort]{natbib}
% tworzenie listingów
\usepackage{listings}
% tworzenie figur wewnątrz figur
\usepackage{subfig}
% do automatycznego skracania nazw rozdziałów i podrozdziałów używanych w nagłówkach strony by mieściły się w jednej linii
\usepackage[fit]{truncate}
% fancyhdr - ładne nagłówki, definicja wyglądu nagłówka, numery stron będą umieszczane w nagłówku po odpowiedniej stronie
\usepackage{fancyhdr}
\pagestyle{fancy}
\renewcommand{\chaptermark}[1]{\markboth{#1}{}}
\renewcommand{\sectionmark}[1]{\markright{\thesection\ #1}}



\fancyhf{}
\fancyhead[LE,RO]{\bfseries\thepage}
% tutaj ograniczamy szerokość pola w nagłówku zawierającego nazwę rozdziału/podrozdziału do 95% szerokości strony
% redefinicja sposobu prezentacji nazw domyślnie wypisywanych wielkimi literami (np. domyślnie w nagłówku Spis treści będzie miał postać SPIS TREŚCI)
% Uwaga! to może popsuć wielkie litery w ogóle! Jak coś nie działa należy usunąć \nouppercase{} z poniższych definicji
\fancyhead[LO]{\nouppercase{\bfseries{\truncate{.95\headwidth}{\rightmark}}}}
\fancyhead[RE]{\nouppercase{\bfseries{\truncate{.95\headwidth}{\leftmark}}}}
\renewcommand{\headrulewidth}{0.5pt}
\renewcommand{\footrulewidth}{0pt}

% definicja typu prostego wymagana przez pierwsze strony rozdziałów itp.
% powyższe reguły niestety tych stron nie dotyczą, gdyż Latex automatycznie przełącza je pomiędzy fancy a plain
% w tym wypadku eliminujemy nagłówki i stopki na stronach początkowych
\fancypagestyle{plain}{%
 \fancyhead{}
 \fancyfoot{}
 \renewcommand{\headrulewidth}{0pt}
 \renewcommand{\footrulewidth}{0pt}
}

\parskip 0.05in


% makro umożliwiające otaczanie symboli okręgami
\usepackage{tikz}
% brak justowania tekstu (bazą okręgu będzie linia tekstu)
\newcommand*\mycirc[1]{%
  \begin{tikzpicture}
    \node[draw,circle,inner sep=1pt] {#1};
  \end{tikzpicture}}

% pionowe justowanie tekstu, środek okręgu pokrywa się ze środkiem tekstu
\newcommand*\mycircalign[1]{%
  \begin{tikzpicture}[baseline=(C.base)]
    \node[draw,circle,inner sep=1pt](C) {#1};
  \end{tikzpicture}}

% zmiana nazwy twierdzeń i lematów
\newtheorem{theorem}{Twierdzenie}[section]
\newtheorem{lemma}[theorem]{Lemat}

% tworzenie definicji dowodu
\newenvironment{proof}[1][Dowód]{\begin{trivlist}
\item[\hskip \labelsep {\bfseries #1}]}{\end{trivlist}}
% \newenvironment{definition}[1][Definicja]{\begin{trivlist}
% \item[\hskip \labelsep {\bfseries #1}]}{\end{trivlist}}
% \newenvironment{example}[1][Przykład]{\begin{trivlist}
% \item[\hskip \labelsep {\bfseries #1}]}{\end{trivlist}}
% \newenvironment{remark}[1][Uwaga]{\begin{trivlist}
% \item[\hskip \labelsep {\bfseries #1}]}{\end{trivlist}}

% definicja czarnego prostokąta zwyczajowo dodawanego na koniec dowodu
\newcommand{\qed}{\nobreak \ifvmode \relax \else
      \ifdim\lastskip<1.5em \hskip-\lastskip
      \hskip1.5em plus0em minus0.5em \fi \nobreak
      \vrule height0.75em width0.5em depth0.25em\fi}

% poniższymi instrukcjami można sterować co ma być numerowane a co nie i co ma być wyświetlane w spisie treści
% \setcounter{secnumdepth}{3}
% \setcounter{tocdepth}{5}

% definicja czcionki mniejszej niż tiny (domyślnie takiej małej nie ma)
\usepackage{lmodern}
\makeatletter
  \newcommand\tinyv{\@setfontsize\tinyv{4pt}{6}}
\makeatother

% definicja jeszcze mniejszej czcionki
\usepackage{lmodern}
\makeatletter
  \newcommand\tinyvv{\@setfontsize\tinyvv{3.5pt}{6}}
\makeatother

% pakiet do obsługi wielostronicowych tabel
\usepackage{longtable}
\setlength{\LTcapwidth}{\textwidth}

\usepackage[section] {placeins}


\usepackage{multirow}

\usepackage{slantsc}
\usepackage[labelsep=endash]{caption}
\addto\captionspolish{\renewcommand{\figurename}{Rys.}}
\addto\captionspolish{\renewcommand{\tablename}{Tab.}}
\addto\captionspolish{\renewcommand*{\appendixpagename}{Dodatki}}
\addto\captionspolish{\renewcommand*{\appendixtocname}{Dodatki}}
\addto\captionspolish{\renewcommand*{\appendixname}{Dodatek}}

\setcounter{secnumdepth}{5}
\usepackage{listings}
\usepackage[toc,page]{appendix}
\usepackage{color}
 
\definecolor{codegreen}{rgb}{0,0.6,0}
\definecolor{codegray}{rgb}{0.5,0.5,0.5}
\definecolor{codepurple}{rgb}{0.58,0,0.82}
\definecolor{backcolour}{rgb}{0.95,0.95,0.92}
 
\lstdefinestyle{mystyle}{
    backgroundcolor=\color{backcolour},   
    commentstyle=\color{codegreen},
    keywordstyle=\color{magenta},
    numberstyle=\tiny\color{codegray},
    stringstyle=\color{codepurple},
    basicstyle=\footnotesize,
    breakatwhitespace=false,         
    breaklines=true,                 
    captionpos=b,                    
    keepspaces=true,                 
    numbers=left,                    
    numbersep=5pt,                  
    showspaces=false,                
    showstringspaces=false,
    showtabs=false,                  
    tabsize=2
}
 
\lstset{style=mystyle}

\begin{document}
\chapter{Testy i wyniki}
W poniższym rozdziale przedstawiono opis testów dokonanych w celu weryfikacji oraz walidacji wykonanej aplikacji. Zamieszczono również wyniki oraz wnioski z nich wynikające.
\section{Testy}
Do zagadnień testowanych należą wymienione w projektach testów ~\ref{sec:testsP} prędkość korzystania z klawiatury ekranowej, przydatność jej udogodnien oraz subiektywne odczucia użytkowników. Ten rodzaj testów zalicza się do akceptacyjnych typu UAT(User Acceptance Tests). 

Przebieg testów następował zgodnie z kolejnością opisaną w projekcie ~\ref{sec:testsP}.
\subsection{Grupa badawcza}
W skład grupy badawczej wchodziły osoby z przedziału wiekowego 21-60 lat, o różnych płciach. Żadna z osób nie była chorą na zanikowe stwardnienie boczne, czy też w inny sposób upośledzona.
\subsection{Warunki początkowe}
Wszystkie osoby podczas testów korzystały z tej samej stacji roboczej i myszki, tak, że środowisko pozostało niezmienione. Przykładowe teksty zawsze były identyczne - wydrukowano je i proszono o ich przepisywanie z kartki. Podczas badań nie brano pod uwagę samopoczucia osób badanych np. stopnia ich zmęcznia, który mógł wpłynąć na ich wyniki. 
\section{Wyniki}
W pierwszej kolejności dokonano testów aplikacji bez dynamicznej zmiany czasów dla progowej wartości czasu fiksacji równej 1,5s. Wyniki pomierzonych, za pomocą stopera, czasów zanotowano w tabeli ~\ref{table:timeMes}. Skróty nazw kolejnych kolumn oznaczają: \textbf{Tk1}- czas dla pisania tekstu pierwszego klawiaturą komputerową,\textbf{Te1}-czas dla pisania tekstu pierwszego klawiaturą ekranową,\textbf{H1}-ilość wykorzystanych autopodpowiedzi podczas pisania tekstu pierwszego, \textbf{Tk2}-czas dla pisania tekstu drugiego klawiaturą komputerową, \textbf{Te2}-czas dla pisania tekstu drugiego klawiaturą ekranową,\textbf{H2}-ilość wykorzystanych autopodpowiedzi podczas pisania tekstu drugiego,\textbf{Ted}-czas pisania tekstu dowolnego klawiaturą ekranową,\textbf{CC}-ilość znaków użytych w dowolnym tekście, \textbf{H3}-ilość wykorzystanych autopodpowiedzi podczas pisania tekstu dowolnego . Ilość znaków w tekście pierwszym jest równa \textbf{142}, natomiast dla tekstu drugiego to \textbf{105} znaków.
\begin{table}
\renewcommand\arraystretch{1.5}
 \centering
    \begin{tabular}{|>{\centering\arraybackslash}m{.5cm}|>{\centering\arraybackslash}m{.5cm}|>{\centering\arraybackslash}m{.5cm}|>{\centering\arraybackslash}m{.5cm}||>{\centering\arraybackslash}m{.5cm}|>{\centering\arraybackslash}m{.5cm}|>{\centering\arraybackslash}m{.5cm}||>{\centering\arraybackslash}m{.5cm}|>{\centering\arraybackslash}m{.5cm}|>{\centering\arraybackslash}m{.5cm}|}
     \hline
    
    \textbf{Lp.} & \textbf{Tk1 [s]}&\textbf{Te1 [s]}& \textbf{H1}&\textbf{Tk2 [s]}& \textbf{Te2 [s]}&\textbf{H2}&\textbf{Ted [s]}&\textbf{CC} &\textbf{H3}\\ \hline   \hline
    \textbf{1}&59&417&2&49&329&1&91&37&1\\\hline
    \textbf{2}&98&548&1&40&385&0&105&42&1\\\hline
    \textbf{3}&112&721&2&80&453&2&172&55&0\\\hline
    \textbf{4}&36&439&1&34&362&1&210&61&1\\\hline
    \textbf{5}&101&447&2&89&355&1&188&59&0\\\hline
    \textbf{6}&123&932&1&96&768&0&186&45&0\\\hline
 \end{tabular}
	 \caption{Tabela wyników pomiarów dla aplikacji bez dynamicznej zmiany czasu.} 
    \label{table:timeMes}
\end{table}
Po dodaniu funkcjonalności dynamicznej zmiany czasu progowego fiksacji zwzroku powtórzono pomiary dla 3 osób badanych. Celem tego zabiego było sprawdzenie, czy dynamiczna zmiana czasu przyspiesza wprowadzanie tekstu na klawiaturze. Wyniki przedstawiono w tabeli ~\ref{table:timeMesSp}.
\begin{table}
\renewcommand\arraystretch{1.5}
 \centering
    \begin{tabular}{|>{\centering\arraybackslash}m{.5cm}|>{\centering\arraybackslash}m{.5cm}|>{\centering\arraybackslash}m{.5cm}|>{\centering\arraybackslash}m{.5cm}||>{\centering\arraybackslash}m{.5cm}|>{\centering\arraybackslash}m{.5cm}|>{\centering\arraybackslash}m{.5cm}||>{\centering\arraybackslash}m{.5cm}|>{\centering\arraybackslash}m{.5cm}|>{\centering\arraybackslash}m{.5cm}|}
     \hline
         \textbf{Lp.} & \textbf{Tk1 [s]}&\textbf{Te1 [s]}& \textbf{H1}&\textbf{Tk2 [s]}& \textbf{Te2 [s]}&\textbf{H2}&\textbf{Ted [s]}&\textbf{CC} &\textbf{H3}\\ \hline   \hline
        \textbf{1}&34&445&2&36&336&1&171&53&0\\\hline
        \textbf{2}&80&548&1&66&415&0&124&36&0\\\hline
        \textbf{3}&42&364&2&25&266&2&92&46&1\\\hline
     \end{tabular}
	 \caption{Tabela wyników pomiarów dla aplikacji z dynamiczną zmianą czasu.} 
    \label{table:timeMesSp}
\end{table}
Dla pomierzonych danych policzono prędkości wpisywania znaków, a wyniki przedstawiono odpowiednio w tabelach ~\ref{table:speed},~\ref{table:speedDyn}. Skróty nazw kolumn oznaczają: \textbf{Vk1}-ilość znaków na sekundę podczas pisania na klawiaturze komputerowej tekstu pierwszego,\textbf{Ve1}-ilość znaków na sekundę podczas pisania na ekranowej komputerowej tekstu pierwszego, \textbf{Vk2}-ilość znaków na sekundę podczas pisania na klawiaturze komputerowej tekstu drugiego, \textbf{Ve2}-ilość znaków na sekundę podczas pisania na klawiaturze ekranowej tekstu drugiego, \textbf{Ved}-ilość znaków na sekundę podczas pisania na klawiaturze ekranowej tekstu dowolnego. Wyniki zaokrąglone zostały do części dziesiętnej.


\begin{table}
\renewcommand\arraystretch{1.5}
 \centering
    \begin{tabular}{|>{\centering\arraybackslash}m{0.5cm}|>{\centering\arraybackslash}m{1.8cm}|>{\centering\arraybackslash}m{1.8cm}||>{\centering\arraybackslash}m{1.8cm}|>{\centering\arraybackslash}m{1.8cm}||>{\centering\arraybackslash}m{1.8cm}|}
     \hline
         \textbf{Lp.} & \textbf{Vk1 [znaków/s]}&\textbf{Ve1 [znaków/s]}& \textbf{Vk2 [znaków/s]}&\textbf{Ve2 [znaków/s]}&\textbf{Veb [znaków/s]}\\ \hline   \hline
        \textbf{1}&2,4&0,3&2,1&0,3&0,4\\\hline
        \textbf{2}&1,5&0,3&2,6&0,3&0,4\\\hline
        \textbf{3}&1,3&0,2&1,3&0,2&0,3\\\hline
        \textbf{4}&3,9&0,3&3,1&0,3&0,3\\\hline
        \textbf{5}&1,4&0,3&1,2&0,3&0,3\\\hline
        \textbf{6}&1,2&0,2&1,1&0,1&0,2\\\hline
     \end{tabular}
	 \caption{Tabela prędkości wpisywania znaków dla aplikacji bez dynamicznej zmiany czasu.} 
    \label{table:speed}
\end{table}
\begin{table}
\renewcommand\arraystretch{1.5}
 \centering
      \begin{tabular}{|>{\centering\arraybackslash}m{0.5cm}|>{\centering\arraybackslash}m{1.8cm}|>{\centering\arraybackslash}m{1.8cm}||>{\centering\arraybackslash}m{1.8cm}|>{\centering\arraybackslash}m{1.8cm}||>{\centering\arraybackslash}m{1.8cm}|}
     \hline
         \textbf{Lp.} & \textbf{Vk1 [znaków/s]}&\textbf{Ve1 [znaków/s]}& \textbf{Vk2 [znaków/s]}&\textbf{Ve2 [znaków/s]}&\textbf{Veb [znaków/s]}\\ \hline   \hline
        \textbf{1}&4,2&0,3&2,9&0,3&0,3\\\hline
        \textbf{2}&1,8&0,3&1,6&0,3&0,3\\\hline
        \textbf{3}&3,4&0,4&4,2&0,4&0,5\\\hline
     \end{tabular}
	 \caption{Tabela prędkości wpisywania znaków dla aplikacji z dynamiczną zmianą czasu.} 
    \label{table:speedDyn}
\end{table}
W celu lepszego porównania otrzymanych wartości prędkości zestawiono ze sobą wartości średnich prędkości dla aplikacji z oraz bez dynamicznej zmiany czasu. Wyniki przedstawiono w tabeli ~\ref{table:avSpeed}. Skróty wierszy to: \textbf{Śr.Vet1 ND }-średnia prędkość dla tekstu numer 1 pisanego na klawiaturze bez dynamicznej zmiany czasu, \textbf{Śr.Vet1 D}-średnia prędkość dla tekstu numer 1 pisanego na klawiaturze z dynamiczną zmianą czasu, \textbf{Śr. Vet2 ND}- średnia prędkość dla tekstu numer 2 pisanego na klawiaturze bez dynamicznej zmiany czasu, \textbf{Śr.Vet2 D}-średnia prędkość dla tekstu numer 2 pisanego na klawiaturze z dynamiczną zmianą czasu, \textbf{Śr. Ved ND}-średnia prędkość dla tekstu dowolnego pisanego na klawiaturze bez dynamicznej zmiany czasu, \textbf{Śr. Ved D}-średnia prędkość dla tekstu dowolnego pisanego na klawiaturze z dynamiczną zmianą czasu. Wyniki średnich zaokrąglono do części setnych.
\begin{table}
\renewcommand\arraystretch{1.5}
 \centering
      \begin{tabular}{|>{\centering\arraybackslash}m{2.5cm}|>{\centering\arraybackslash}m{2.5cm}|}
     \hline
         \textbf{Śr. Vet1 ND [znaków/s]} &0,26\\\hline
          \textbf{Śr. Vet1 D [znaków/s]}&0,32\\\hline
          \hline
          \textbf{Śr. Vet2 ND [znaków/s]}& 0,25\\\hline
          \textbf{Śr. Vet2 D [znaków/s]}&0,32\\\hline
          \hline
          \textbf{Śr. Ved ND [znaków/s]}&0,33\\\hline
           \textbf{Śr. Ved D [znaków/s]}&0,37\\ \hline   
       
     \end{tabular}
	 \caption{Tabela średnich prędkości wpisywania znaków dla aplikacji} 
    \label{table:avSpeed}
\end{table}
Policzono również średnie sumaryczne dla: prędkości wpisywania znaków poprzez klawiaturę ekranową - \textbf{2,28 [znaków/s]}, prędkości wpisywania znaków poprzez klawiaturę ekranową bez dynamicznej zmiany czasu - \textbf{0,26 [znaków/s]} oraz prędkości wpisywania znaków poprzez klawiaturę ekranową z dynamiczną zmianą czasu -\textbf{0,32 [znaków/s]}.
W wyniku rozmowy podsumowującej z każdym z uczestników badania wynotowano następujące uwagi i wnioski na temat pracy z klawiaturą ekranową: \label{opinion}
\begin{itemize}
\item Podpowiedzi są mało przydante ze wględu na to, że przedstawiają 4 pierwsze słowa w kolejności alfabetycznej, a nie 4 pierwsze słowa ze względu na częstość występowania w języku Polskim.
\item Zbyt krótki czas progowy fiksacji prowadzi do dużej ilości błedów i wpływa irytująco na użytkownika. 
\item Najchętniej wybieranym zestawem kolorystycznym jest \textit{Modern}.
\item Pierwsze próby pracy z klawiaturą ekarnową są wysoce męczące, jednak przy kolejnych próbach szybko widać postępy w sposobie korzystania z niej i praca staje się przyjemniejsza.
\item Wygląd przycisków \textit{CapsLock} oraz \textit{Shift} mógłby być bardziej jednoznaczny. Często dochodzi do pomyłek.
\end{itemize} 
\subsection{Wnioski}
Analizując i porównując powyżej przedstawione wyniki pomiarowe moża nauważyć następujące poprawności:
\begin{itemize}
\item Prędkość wpisywania tekstu klawiaturą ekranową jest średnio 8,8 razy mniejsza dla klawiatury bez dynamicznej zmiany czasu oraz 7,1 razy mniejsza dla klawiatury z dynamiczną zmianą czasu. 
\item Prędkość wpisywania znaków na klawiaturze ekranowej jest niezależna od tekstu wprowadzanego i waha się między 0,1 a 0,5 znaków na sekundę. 
\item Średnia wartość prędkości dla tekstu dowolnego jest nawet do 0,12 znaku na sekundę większa od prędkości dla tekstu przepisywanego.
\item Przyciski podpowiedzi są częściej używane, gdy tekst był już wcześniej znany i przepisywany, ale rzadziej dla tekstu dowolnego.
\item Przyciski podpowiedzi nie przyspieszają wpisywania tekstu.
\item Różnice prędkości średnich dla klawiatury bez i z dynamiczną zmianą czasu dla każdego testu są znikome, jednak zauważa się niewielkie zwiększenie prędkości dla klawiatury z dynamiczną zmianą czasu. 
\end{itemize}
Na podstawie powyższych wniosków można stwierdzić, iż w celu poprawy jakości działania klawiatury ekranowej należy dopracować algorytm dynamicznej zmiany czasu oraz zamienić algorytm tworzący autouzupełnienia słów, tak by ten brał pod uwagę częstość wykorzystania daniego słowa zamiast jego alfabetyczną kolejność.
\end{document}